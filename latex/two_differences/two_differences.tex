\documentclass[12pt]{article}
\usepackage{physics}
\usepackage{breqn}
\title{FCI Questions}
\author{Patryk Kozlowski}
\date{\today}
\begin{document}
\maketitle
two differences
\begin{equation}
    \mel{\Psi }{V}{\Psi (k\rightarrow k^{\prime}, l\rightarrow l^{\prime})}
    =[mp|nq]-[mq|np]
\end{equation}
\begin{equation}
    =(mp|nq)\delta _{[m][p]}\delta _{[n][q]}-(mq|np)\delta _{[m][q]}\delta _{[n][p]}
\end{equation}
 there are 2, I think, cases to go from from here. let's consider that the shared orbs are \{a,b,c,d\} with the differences orbs an determinant 1 being \{m,n\} and in determinant 2 being \{p,q\}. m,n,and p,q can all have the same spin, or the unique herbs in each determinant can be of different spin, like m and n being in a combination of alpha and beater. I read the part where you said that second quantization already takes into account the indistinguishability of fermions, but I don't understand how that could manifest itself here. we can't just label m and n by indices and setting m being alpha and n being beta, because they are in indistinguishable particles. thought is that the same should go for p and q. apart from indistinguishability matters of fermions, my understanding is that because h6 is not an ionic species, the number of alpha herbs needs to equal the number of beta herbs. this means, that we can't have m and n with same spin, and p and q with the same, but opposite spin, as this would violate the fact that the spin of a total valid determinant has to be 0 and we cannot choose the common orbs \{a,b,c,d\} to have a spin that would offset the spins of the  BOTH sets of unique herbs, which could be 1 or negative one. in other words, for for a generic determined \{a,b,c,d,e,f\} need, $N_{\alpha }=N_{\beta }$, so \{m,n\} and \{p,q\} cannot be of different spin, so m,n and p,q has both spin alpha/beater, OR m and n and p and q BOTH having different spins so that the combined spin of the pair is 0. once again, I will bring up indistinguishability here: if m and n have diff spin, we cannot just assign specific spin indices to each them, because they are indistinguishable particles. how does second quantization automatically take this into account? 
\section{Case 1}
All 2x2=4 (m,n,p,q)unique orbs can be of the same spin.
\begin{equation}
    (mp|nq)\delta _{[m][p]}\delta _{[n][q]}-(mq|np)\delta _{[m][q]}\delta _{[n][p]}
\end{equation}
\begin{equation}
    =(mp|nq)-(mq|np)
\end{equation}
\section{case 2}
\subsection{the unique spin orbs in each determinant are of different spins.}

this boils down to find out the manifestation of something like $\delta _{[m][p]}\delta _{[n][q]}$. this is where I reference the superposition principle, because I don't see how the deltas can be evaluated otherwise, since m,n,p,q have indefinite spin, so my thought Was that $\delta _{[m][p]}$ is sometimes equal to 0 and sometimes equal to 1, so we give it a value of 1/2.
\begin{equation}
    [mp|nq]-[mq|np]
\end{equation}
\begin{equation}
    =(1/2)*(1/2)*(mp|nq)-(1/2)*(1/2)*(mq|np)
\end{equation}
although I do understand that we are dealing with matrix elements, I don't see how I can evaluate the delta functions without invoking something like the superposition principle.
\end{document}