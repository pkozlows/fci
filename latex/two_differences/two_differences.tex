\documentclass[12pt]{article}
\usepackage{physics}
\usepackage{breqn}
\title{FCI Questions}
\author{Patryk Kozlowski}
\date{\today}
\begin{document}
\maketitle
\section{two differences}
\begin{equation}
    \mel{\Psi }{V}{\Psi (k\rightarrow k^{\prime}, l\rightarrow l^{\prime})}=v^{\alpha\beta\gamma\delta}\left(-1\right)^{\varepsilon\left(\kappa_{1},\dots,\kappa_{i}^{\prime},\dots,\kappa _{j},...,\kappa_{n}\right)}\\
\end{equation}
\begin{equation}
    \bra{0}\left(\prod_{\kappa=\left(\kappa_{n}\dots\kappa_{3}\right)}a_{\kappa}\right)
        a_{2}a_{1}a^{\dag}_{\alpha }a^{\dag}_{\beta }a_{\gamma }a_{\delta }a^{\dag}_{1^{\prime}}a^{\dag}_{2^{\prime}}
    \left(\prod_{\kappa^{\prime}=\left(\kappa_{3}\dots\kappa_{n}\right)}a^{\dag}_{\kappa^{\prime}}\right)\ket{0}
\end{equation}
% \begin{equation}
%     =\bra{0}\left(\prod_{\kappa=\left(\kappa_{n}\dots\kappa_{3}\right)}a_{\kappa}\right)
%         a_{2}a_{1}a^{\dag}_{\alpha }a^{\dag}_{\beta }a_{\gamma }\delta _{\delta  \kappa _{1^{\prime}}}a^{\dag}_{2^{\prime}}
%     \left(\prod_{\kappa^{\prime}=\left(\kappa_{3}\dots\kappa_{n}\right)}a^{\dag}_{\kappa^{\prime}}\right)\ket{0}
% \end{equation}
% \begin{equation}
%     -\bra{0}\left(\prod_{\kappa=\left(\kappa_{n}\dots\kappa_{3}\right)}a_{\kappa}\right)
%         a_{2}a_{1}a^{\dag}_{\alpha }a^{\dag}_{\beta }a_{\gamma }a^{\dag}_{1^{\prime}}a_{\delta }a^{\dag}_{2^{\prime}}
%     \left(\prod_{\kappa^{\prime}=\left(\kappa_{3}\dots\kappa_{n}\right)}a^{\dag}_{\kappa^{\prime}}\right)\ket{0}
% \end{equation}
% \begin{equation}
%     =\bra{0}\left(\prod_{\kappa=\left(\kappa_{n}\dots\kappa_{3}\right)}a_{\kappa}\right)
%         a_{2}a_{1}a^{\dag}_{\alpha }a^{\dag}_{\beta }\delta _{\gamma  \kappa _{2^{\prime}}}\delta _{\delta  \kappa _{1^{\prime}}}
%     \left(\prod_{\kappa^{\prime}=\left(\kappa_{3}\dots\kappa_{n}\right)}a^{\dag}_{\kappa^{\prime}}\right)\ket{0}
% \end{equation}
% \begin{equation}
%     -0
% \end{equation}
% \begin{equation}
%     -\bra{0}\left(\prod_{\kappa=\left(\kappa_{n}\dots\kappa_{3}\right)}a_{\kappa}\right)
%         a_{2}a_{1}a^{\dag}_{\alpha }a^{\dag}_{\beta }\delta _{\gamma  \kappa _{1^{\prime}}}a_{\delta }a^{\dag}_{2^{\prime}}
%     \left(\prod_{\kappa^{\prime}=\left(\kappa_{3}\dots\kappa_{n}\right)}a^{\dag}_{\kappa^{\prime}}\right)\ket{0}
% \end{equation}
% \begin{equation}
%     +0
% \end{equation}
% \begin{equation}
%     =\delta _{\alpha  \kappa _{1}}\delta _{\beta  \kappa _{2}}\delta _{\gamma  \kappa _{2^{\prime}}}\delta _{\delta  \kappa _{1^{\prime}}}
% \end{equation}
% \begin{equation}
%     -\delta _{\alpha  \kappa _{1}}\delta _{\beta  \kappa _{2}}\delta _{\gamma  \kappa _{1^{\prime}}}\delta _{\delta  \kappa _{2^{\prime}}}
% \end{equation}
% \begin{equation}
%     =v^{122^{\prime}1^{\prime}}-v^{121^{\prime}2^{\prime}}
% \end{equation}
\begin{equation}
    =[mp|nq]-[mq|np]
\end{equation}
\begin{equation}
    =(mp|nq)\delta _{[m][p]}\delta _{[n][q]}-(mq|np)\delta _{[m][q]}\delta _{[n][p]}
\end{equation}
% The total spent in the excitation must be preserved, so either m n, p q all alpha, m n, p q all beta, or m n, p q have one of each. 
% $\delta _{[m][p]}\delta _{[n][q]}=1$ iff $m=p$ and $n=q$. total number of possible double excitations is 6 total for m and then 5 choices for n. same for p,q so 30**2. for m n, p q all alpha have 3 for m, 2 for n then 3 for p, 2 for q, so 6**2=36. for m n, p q have one of each have 3 for m, 2 for n then 3 for p, 2 for q, so 6**2=36 cousin. for m n, p q have one of each 3 for m, 3 for n, 3 for p, 3 for q so 3**4=81.
\end{document}