\documentclass[12pt]{article}
\usepackage{physics}
\usepackage{breqn}
\title{FCI Questions}
\author{Patryk Kozlowski}
\date{\today}
\begin{document}
\maketitle
two differences
\begin{equation}
    \mel{\Psi }{V}{\Psi (k\rightarrow k^{\prime}, l\rightarrow l^{\prime})}
    =[mp|nq]-[mq|np]
\end{equation}
\begin{equation}
    =(mp|nq)\delta _{[m][p]}\delta _{[n][q]}-(mq|np)\delta _{[m][q]}\delta _{[n][p]}
\end{equation}
 there are a few cases to go from from here.
\section{Case 1}
All 2x2=4 (m,n,p,q)so not unique orbs can be of the same spin.
\begin{equation}
    =(mp|nq)-(mq|np)
\end{equation}
\section{case 2}
\subsection{the unique spin orbs in each determinant are of different spins.}

The fermions we are dealing with are indistinguishable, so I need to find out the manifestation of something like $\delta _{[m][p]}\delta _{[n][q]}$ where I have the differences between the determinants stored in the tuple like ([m,n], [p,q]). I've recognized that since the particles are indistinguishable the differences should actually be stored inset like (\{m,n\}, \{p,q\}) but if I'm understanding correctly this difference just makes my program slower because I'm not taking advantage of the inherent symmetry and shouldn't really affect the energetics, so I don't have to worry about changing it for now.
So, $\delta _{[m][p]}$ is in the super position of being unity or 0, if the spins might be the same or different, so in this case I would set $\delta _{[m][p]}=1/2$. so with this modification, my derivation should look like
\begin{equation}
    [mp|nq]-[mq|np]
\end{equation}
\begin{equation}
    =(1/2)*(1/2)*(mp|nq)-(1/2)*(1/2)*(mq|np)
\end{equation}
however, theoretically this isn't giving me a correct energy when I implemented it into my program so I am very confused like garnet (lol). I hope this makes sense this time. what do I need to think about to get the theoretics correct here?
\end{document}