\documentclass[12pt]{article}
\usepackage{physics}
\usepackage{breqn}
\title{FCI Questions}
\author{Patryk Kozlowski}
\date{\today}
\begin{document}
\maketitle
two differences
\begin{equation}
    \mel{\Psi }{V}{\Psi (k\rightarrow k^{\prime}, l\rightarrow l^{\prime})}
    =[mp|nq]-[mq|np]
\end{equation}
\begin{equation}
    =(mp|nq)\delta _{[m][p]}\delta _{[n][q]}-(mq|np)\delta _{[m][q]}\delta _{[n][p]}
\end{equation}
 there are a few cases to go from from here.
\section{Case 1}
All 2x2=4 (m,n,p,q)so not unique orbs can be of the same spin.
\begin{equation}
    =(mp|nq)-(mq|np)
\end{equation}
\section{case 2}
the unique spin orbs in each determinant are of different spins.
My thought is that because we're working in quantum mechanics we will be dealing with expectation values. So, $|\delta _{[m][p]}\delta _{[n][q]}|=0.5$ and $|\delta _{[m][q]}\delta _{[n][p]}|=0.5$. So, 
\begin{equation}
    [mp|nq]-[mq|np]
\end{equation}
\begin{equation}
    =(1/2)*(mp|nq)-(1/2)*(mq|np)
\end{equation}
previously I wasn't thinking about the too electron different case in terms of different cases, like I do here.however my energy went from aomething like -7.83, so close to correct, to -7.75, so not so close.I am confused because initially I was not treating the two electron difference case correctly in terms of theoretics(or so I think), but now I am, and my energy is getting further away. can you give any hints, or should I just spend some more time thinking about this?
% The total spent in the excitation must be preserved, so either m n, p q all alpha, m n, p q all beta, or m n, p q have one of each. 
% $\delta _{[m][p]}\delta _{[n][q]}=1$ iff $m=p$ and $n=q$. total number of possible double excitations is 6 total for m and then 5 choices for n. same for p,q so 30**2. for m n, p q all alpha have 3 for m, 2 for n then 3 for p, 2 for q, so 6**2=36. for m n, p q have one of each have 3 for m, 2 for n then 3 for p, 2 for q, so 6**2=36 cousin. for m n, p q have one of each 3 for m, 3 for n, 3 for p, 3 for q so 3**4=81.
\end{document}