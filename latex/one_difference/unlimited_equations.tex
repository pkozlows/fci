\documentclass{article}
\usepackage{amsmath}

\newcounter{subeqn}
\newcounter{parentequation}

\makeatletter
\newenvironment{unlimitedsubequations}{%
  \refstepcounter{equation}%
  \protected@edef\theparentequation{\theequation}%
  \setcounter{subeqn}{0}%
  \def\theequation{\theparentequation\alph{subeqn}}%
  \ignorespaces
}{%
  \setcounter{equation}{\value{parentequation}}%
  \ignorespacesafterend
}
\makeatother

\begin{document}

\section{Example}

\begin{equation}
  \label{eq:parent}
  A = B + C
\end{equation}

\begin{unlimitedsubequations}
  \begin{equation}
    \label{eq:sub1}
    B = D + E
  \end{equation}
  \begin{unlimitedsubequations}
    \begin{equation}
      \label{eq:sub2}
      D = F + G
    \end{equation}
    \begin{equation}
      \label{eq:sub3}
      E = H + I
    \end{equation}
  \end{unlimitedsubequations}
\end{unlimitedsubequations}

Equation \eqref{eq:parent} has subequations \eqref{eq:sub1}, \eqref{eq:sub2}, and \eqref{eq:sub3}.

\end{document}


